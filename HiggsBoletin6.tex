\documentclass[12pt]{article}

%\usepackage[spanish]{babel}
\usepackage{amsmath,latexsym,amssymb}

\usepackage{graphicx}

\usepackage{geometry}
%\usepackage{showframe} %This line can be used to clearly show the new margins
%\newgeometry{vmargin={15mm}, hmargin={12mm,17mm}}   % set the margins
%\newgeometry{vmargin={20mm}, hmargin={20mm,20mm}}   % set the margins
\newgeometry{vmargin={20mm}, hmargin={20mm,20mm}}

\newcommand{\R}{\mathbb{R}}

\usepackage{subcaption}

%\newcommand{\aa}{\'{a}}
%\newcommand{\'i}{\'{\i}}
%\newcommand{\o}{\'{o}}

\begin{document}

\begin{center}

{\Large{\underline{\bf En memoria de Peter Higgs (1929 -- 2024)}}} \vspace*{2mm} \vspace*{8mm} \\

{\small Jos\'e Antonio Garc\'ia-Hern\'andez y Wolfgang Bietenholz \\
  Instituto de Ciencias Nucleares \\
  Universidad Nacional Aut\'onoma de M\'exico (UNAM) \\
Apartado Postal 70-543, 04510 Ciudad de M\'exico, M\'exico}
 \vspace*{6mm} \\

\end{center}

\noindent
Peter Higgs fue un f\'isico te\'orico brit\'anico, famoso por su
trabajo de 1964 donde propuso un mecanismo que puede generar masas
para part\'iculas elementales, conforme a la simetr\'{\i}a de norma.
Medio siglo despu\'es, dos experimentos del CERN confirmaron
que este mecanismo est\'a realizado en la naturaleza.
El 8 de abril nos lleg\'{o} la triste noticia del fallecimiento del
gran pionero de la f\'{\i}sica de part\'{\i}culas elementales.
Este art\'{\i}culo es dedicado a su memoria, al mecanismo
y la part\'icula que llevan su nombre.

\section{Datos biogr\'aficos y contexto hist\'orico}

\input{sec1c}

\section{Mecanismo de Higgs}

\input{sec2c}

\noindent
{\bf Agradecimiento:} Este trabajo fue apoyado por UNAM-DGAPA
con el proyecto del PAPIIT IG100322, y por el Consejo Nacional de
Humanidades, Ciencia y Tecnología (CONAHCYT).

\begin{thebibliography}{10}

\bibitem{Higgs} P.W.\ Higgs,
% Broken symmetries, massless particles and gauge fields,
{\em Phys.\ Lett.}\ {\bf 12} (1964) 132-133;
% doi:10.1016/0031-9163(64)91136-9
% Broken Symmetries and the Masses of Gauge Bosons,
{\em Phys.\ Rev.\ Lett.}\ {\bf 13} (1964) 508-509.
% doi:10.1103/PhysRevLett.13.508
% Spontaneous Symmetry Breakdown without Massless Bosons,
{\em Phys.\ Rev.}\ {\bf 145} (1966) 1156-1163.
% doi:10.1103/PhysRev.145.1156

\bibitem{boson} P.\ Higgs,
% My Life as a Boson: The Story of ``The Higgs'',
{\em Int.\ J.\ Phys.}\ {\bf A17} (2002) 86-88.
% https://doi.org/10.1142/S0217751X02013046

\bibitem{EB} F.\ Englert y R.\ Brout,
% Broken Symmetry and the Mass of Gauge Vector Mesons,
{\em Phys.\ Rev.\ Lett.}\ {\bf 13} (1964) 321-323.
% doi:10.1103/PhysRevLett.13.321

\bibitem{GHK} G.S.\ Guralnik, C.R.\ Hagen y T.W.B.\ Kibble,
% Global Conservation Laws and Massless Particles,
{\em Phys.\ Rev.\ Lett.}\ {\bf 13} (1964) 585-587.
% doi:10.1103/PhysRevLett.13.585

\bibitem{Anderson} P.W.\ Anderson,
% Plasmons, Gauge Invariance, and Mass,
{\it Phys.\ Rev.}\ {\bf 130} (1963) 439-442.
% doi:10.1103/PhysRev.130.439

\bibitem{Schwinger} J.S.\ Schwinger,
% Gauge Invariance and Mass,
  {\em Phys.\ Rev.}\ {\bf 125} (1962) 397-398;
% doi:10.1103/PhysRev.125.397
% Gauge Invariance and Mass. 2.,
{\em Phys.\ Rev.}\ {\bf 128} (1962) 2425-2429.
% doi:10.1103/PhysRev.128.2425

\bibitem{Nobel13} Royal Swedish Academy of Sciences,
  ``Scientific Background: The BEH-Mechanism, Interactions with Short
  Range Forces and Scalar Particles'',\\
  www.nobelprize.org/prizes/physics/2013/advanced-information/
  
\bibitem{MigPol} A.A.\ Migdal y A.M.\ Polyakov,
% SPONTANEOUS BREAKDOWN OF STRONG INTERACTION SYMMETRY
% AND THE ABSENCE OF MASSLESS PARTICLES,
{\em Zh.\ Eksp.\ Teor.\ Fiz.}\ {\bf 51} (1966) 135-146 
[{\em Sov.\ Phys.\ JETP} {\bf 24} (1967) 91-98].

\bibitem{Weinberg} S.\ Weinberg,
% A Model of Leptons
{\em Phys.\ Rev.\ Lett.}\ {\bf 19} (1967) 1264-1266.
% doi:10.1103/PhysRevLett.19.1264  

\bibitem{Salam} A.\ Salam,
% Weak and Electromagnetic Interactions,
  en {\em Proc.\ of the 8th Nobel Symposium on ``Elementary
  particle theory, relativistic groups and analyticity''},
  editor N.\ Svartholm (1968) p.\ 367-377.
% Conf. Proc. C \textbf{680519}, 367-377 (1968)
% doi:10.1142/9789812795915\_0034

\bibitem{Glashow} S.L.\ Glashow,
% Partial Symmetries of Weak Interactions
{\it Nucl.\ Phys.}\ {\bf 22} (1961) 579-588.
% doi:10.1016/0029-5582(61)90469-2

\bibitem{tHooft} G.\ 't Hooft,
% Renormalization of Massless Yang-Mills Fields
{\em Nucl.\ Phys.}\ {\bf B33} (1971) 173-199;
% doi:10.1016/0550-3213(71)90395-6
% Renormalizable Lagrangians for Massive Yang-Mills Fields
{\em Nucl.\ Phys.}\ {\bf B35} (1971) 167-188.
G.\ 't Hooft y M.J.G.\ Veltman,
% doi:10.1016/0550-3213(71)90139-8
% Regularization and Renormalization of Gauge Fields,
{\em Nucl.\ Phys.}\ {\bf B44} (1972) 189-213.
% doi:10.1016/0550-3213(72)90279-9

\bibitem{BolGiam} C.G.\ Bollini y J.J.\ Giambiagi,
% Dimensional Renormalization: The Number of Dimensions
% as a Regularizing Parameter,
{\em Nuovo Cim.}\ {\bf B12} (1972) 20-26;
% doi:10.1007/BF02895558
% Lowest order divergent graphs in nu-dimensional space,
{\em Phys.\ Lett.}\ {\bf B40} (1972) 566-568.
% doi:10.1016/0370-2693(72)90483-2

\bibitem{DimReg} W.\ Bietenholz y L.\ Prado, 
% 40 Years of Calculus in 4+ epsilon Dimensions
%{\em Bolet\'in de la Sociedad Mexicana de F\'isica}
{\em Bol.\ Soc.\ Mex.\ F\'is.}\ {\bf 26-4} (2012) 227-230;
% Revolutionary physics in reactionary Argentina
{\em Physics Today} {\bf 67} (2014) 38-43.
% DOI 10.1063/PT.3.2277

\bibitem{QCD} H.\ Fritzsch, M.\ Gell-Mann y H.\ Leutwyler,
% Advantages of the Color Octet Gluon Picture,
{\em Phys.\ Lett.}\ {\bf B47} (1973) 365-368.
% doi:10.1016/0370-2693(73)90625-4

\bibitem{Lederman} L.M.\ Lederman y D.\ Teresi,
``The God Particle'', Dell Publishing, 1993.

\bibitem{Guardian} Entrevista con Peter Higgs, publicado en
  {\em The Guardian}, Dec. 6, 2013.
  
\bibitem{ATLASCMS} ATLAS Collaboration,
% Observation of a new particle in the search for the Standard Model
% Higgs boson with the ATLAS detector at the LHC
{\em Phys.\ Lett.}\ {\bf B716} (2012) 1-29.
% https://doi.org/10.1016/j.physletb.2012.08.020
CMS Collaboration,
% Observation of a new boson at a mass of 125 GeV with the
% CMS experiment at the LHC".
{\em Phys.\ Lett.}\ {\bf B716} (2012) 30-61.
% doi:10.1016/j.physletb.2012.08.021

\bibitem{neutrinos} A.\ Aguilar-Ar\'{e}valo y W.\ Bietenholz,
% NEUTRINOS: Mysterious Particles with Fascinating
% Features, which led to the Physics Nobel Prize 2015  
{\em Rev.\ Cub.\ F\a'{\i}s.}\ {\bf 32} (2015) 127-136
[versi\'on m\'as extensa: arXiv:1601.04747 [physics.pop-ph]].

\bibitem{HiggsBol} D.\ Ayala Garc\'ia y W.\ Bietenholz,
% Partícula de Higgs: ¿Qué es, y por qué la necesitamos tanto?
% {\em Bolet\'in de la Sociedad Mexicana de F\'isica}
{\em Bol.\ Soc.\ Mex.\ F\'is.}\ {\bf 26-3} (2012) 161-166.  
W.\ Bietenholz,
% The Higgs Particle: what is it, and why did it lead to a
% Nobel Prize in Physics?
{\em Rev. Cub.\ F\'is.}\ {\bf 30} (2013) 109-112.

\bibitem{WBparti}  W.\ Bietenholz,
% What are Elementary Particles? From Dark Energy
% to Quantum Field Excitations
{\em Rev.\ Cub.\ F\a'{\i}s.}\ {\bf 37} (2020) 146-151.
% arXiv:2011.07719 [physics.pop-ph]

\bibitem{Shifman} M.\ Shifman (editor), ``Standing Together in Troubled
  Times: Unpublished Letters by Pauli, Einstein, Franck and Others'',
  World Scientific, 2017.

\bibitem{Nambu} Y.\ Nambu,
% Axial Vector Current Conservation in Weak Interactions
{\em Phys.\ Rev.\ Lett.}\ {\bf 4} (1960) 380-382.
% https://doi.org/10.1103/PhysRevLett.4.380
Y.\ Nambu y G.\ Jona-Lasinio,
% Dynamical Model of Elementary Particles Based on an Analogy
% with Superconductivity. I
{\em Phys.\ Rev.}\ {\bf 122} (1961) 345-358;
% https://doi.org/10.1103/PhysRev.122.345
% Dynamical Model of Elementary Particles Based on an Analogy
% with Superconductivity. II
{\em Phys.\ Rev.}\ {\bf 124} (1961) 246-254.
% https://doi.org/10.1103/PhysRev.124.246

\bibitem{Goldstone} J.\ Goldstone,
% Field Theories with Superconductor Solutions
{\em Nuovo Cim.}\ {\bf 19} (1961) 154-164.
% 10.1007/BF02812722. 
  
\bibitem{GSW} J.\ Goldstone, A.\ Salam y S.\ Weinberg,
% Broken Symmetries
  {\em Phys.\ Rev.}\ {\bf 127} (1962) 965-970.
% https://doi.org/10.1103/PhysRev.127.965

\bibitem{Landau} L.D.\ Landau, 
% On the angular momentum of a system of two photons
{\em Dokl.\ Akad.\ Nauk.\ SSSR} {\bf 60} (1948) 207-209.

\end{thebibliography}

\end{document}