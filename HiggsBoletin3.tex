\documentclass[12pt]{article}

%\usepackage[spanish]{babel}
\usepackage{amsmath,latexsym,amssymb}

\usepackage{graphicx}

\usepackage{geometry}
%\usepackage{showframe} %This line can be used to clearly show the new margins
%\newgeometry{vmargin={15mm}, hmargin={12mm,17mm}}   % set the margins
%\newgeometry{vmargin={20mm}, hmargin={20mm,20mm}}   % set the margins
\newgeometry{vmargin={20mm}, hmargin={20mm,20mm}}

\newcommand{\R}{\mathbb{R}}

\usepackage{subcaption}

%\newcommand{\aa}{\'{a}}
%\newcommand{\'i}{\'{\i}}
%\newcommand{\o}{\'{o}}

\begin{document}

\begin{center}

{\Large{\underline{\bf En memoria de Peter Higgs (1929 -- 2024)}}} \vspace*{2mm} \\

\end{center}

\noindent
Peter Higgs fue un f\'isico te\'orico brit\'anico, famoso por su
trabajo de 1964 donde propuso un mecanismo que puede generar masas
para part\'iculas elementales, conforme a la simetr\'{\i}a de norma.
Medio siglo despu\'es, dos experimentos del CERN confirmaron
que este mecanismo est\'a realizado en la naturaleza.
El 8 de abril nos lleg\'{o} la triste noticia del fallecimiento del
gran pionero de la f\'{\i}sica de part\'{\i}culas elementales.
Este art\'{\i}culo es dedicado a su memoria, al mecanismo
y la part\'icula que llevan su nombre.

\section{Datos biogr\'aficos y contexto hist\'orico}

\input{sec1a}

\section{Mecanismo de Higgs}

\input{sec2a}

\begin{thebibliography}{10}

\bibitem{Higgs} P.W.\ Higgs,
% Broken symmetries, massless particles and gauge fields,
{\em Phys.\ Lett.}\ {\bf 12} (1964) 132-133;
% doi:10.1016/0031-9163(64)91136-9
% Broken Symmetries and the Masses of Gauge Bosons,
{\em Phys.\ Rev.\ Lett.}\ {\bf 13} (1964) 508-509.
% doi:10.1103/PhysRevLett.13.508
% Spontaneous Symmetry Breakdown without Massless Bosons,
{\em Phys.\ Rev.}\ {\bf 145} (1966) 1156-1163.
% doi:10.1103/PhysRev.145.1156

\bibitem{boson} P.\ Higgs,
% My Life as a Boson: The Story of ``The Higgs'',
{\em Int.\ J.\ Phys.}\ {\bf A17} (2002) 86-88.
% https://doi.org/10.1142/S0217751X02013046

\bibitem{EB} F.\ Englert y R.\ Brout,
% Broken Symmetry and the Mass of Gauge Vector Mesons,
{\em Phys.\ Rev.\ Lett.}\ {\bf 13} (1964) 321-323.
% doi:10.1103/PhysRevLett.13.321

\bibitem{GHK} G.S.\ Guralnik, C.R.\ Hagen y T.W.B.\ Kibble,
% Global Conservation Laws and Massless Particles,
{\em Phys.\ Rev.\ Lett.}\ {\bf 13} (1964) 585-587.
% doi:10.1103/PhysRevLett.13.585

\bibitem{Anderson} P.W.\ Anderson,
% Plasmons, Gauge Invariance, and Mass,
{\it Phys.\ Rev.}\ {\bf 130} (1963) 439-442.
% doi:10.1103/PhysRev.130.439

\bibitem{Schwinger} J.S.\ Schwinger,
% Gauge Invariance and Mass,
  {\em Phys.\ Rev.}\ {\bf 125} (1962) 397-398;
% doi:10.1103/PhysRev.125.397
% Gauge Invariance and Mass. 2.,
{\em Phys.\ Rev.}\ {\bf 128} (1962) 2425-2429.
% doi:10.1103/PhysRev.128.2425

\bibitem{Nobel13} Royal Swedish Academy of Sciences,
  ``Scientific Background: The BEH-Mechanism, Interactions with Short
  Range Forces and Scalar Particles'',\\
  www.nobelprize.org/prizes/physics/2013/advanced-information/
  
\bibitem{MigPol} A.A.\ Migdal y A.M.\ Polyakov,
% SPONTANEOUS BREAKDOWN OF STRONG INTERACTION SYMMETRY
% AND THE ABSENCE OF MASSLESS PARTICLES,
{\em Zh.\ Eksp.\ Teor.\ Fiz.}\ {\bf 51} (1966) 135-146 
[{\em Sov.\ Phys.\ JETP} {\bf 24} (1967) 91-98].

\bibitem{Weinberg} S.\ Weinberg,
% A Model of Leptons
{\em Phys.\ Rev.\ Lett.}\ {\bf 19} (1967) 1264-1266.
% doi:10.1103/PhysRevLett.19.1264  

\bibitem{Salam} A.\ Salam,
% Weak and Electromagnetic Interactions,
  en {\em Proc.\ of the 8th Nobel Symposium on ``Elementary
  particle theory, relativistic groups and analyticity''},
  editor N.\ Svartholm (1968) p.\ 367-377.
% Conf. Proc. C \textbf{680519}, 367-377 (1968)
% doi:10.1142/9789812795915\_0034

\bibitem{Glashow} S.L.\ Glashow,
% Partial Symmetries of Weak Interactions
{\it Nucl.\ Phys.}\ {\bf 22} (1961) 579-588.
% doi:10.1016/0029-5582(61)90469-2

\bibitem{tHooft} G.\ 't Hooft,
% Renormalization of Massless Yang-Mills Fields
{\em Nucl.\ Phys.}\ {\bf B33} (1971) 173-199;
% doi:10.1016/0550-3213(71)90395-6
% Renormalizable Lagrangians for Massive Yang-Mills Fields
{\em Nucl.\ Phys.}\ {\bf B35} (1971) 167-188.
G.\ 't Hooft y M.J.G.\ Veltman,
% doi:10.1016/0550-3213(71)90139-8
% Regularization and Renormalization of Gauge Fields,
{\em Nucl.\ Phys.}\ {\bf B44} (1972) 189-213.
% doi:10.1016/0550-3213(72)90279-9

\bibitem{BolGiam} C.G.\ Bollini y J.J.\ Giambiagi,
% Dimensional Renormalization: The Number of Dimensions
% as a Regularizing Parameter,
{\em Nuovo Cim.}\ {\bf B12} (1972) 20-26;
% doi:10.1007/BF02895558
% Lowest order divergent graphs in nu-dimensional space,
{\em Phys.\ Lett.}\ {\bf B40} (1972) 566-568.
% doi:10.1016/0370-2693(72)90483-2

\bibitem{DimReg} W.\ Bietenholz y L.\ Prado, 
% 40 Years of Calculus in 4+ epsilon Dimensions
%{\em Bolet\'in de la Sociedad Mexicana de F\'isica}
{\em Bol.\ Soc.\ Mex.\ F\'is.}\ {\bf 26-4} (2012) 227-230;
% Revolutionary physics in reactionary Argentina
{\em Physics Today} {\bf 67} (2014) 38-43.
% DOI 10.1063/PT.3.2277

\bibitem{QCD} H.\ Fritzsch, M.\ Gell-Mann y H.\ Leutwyler,
% Advantages of the Color Octet Gluon Picture,
{\em Phys.\ Lett.}\ {\bf B47} (1973) 365-368.
% doi:10.1016/0370-2693(73)90625-4

\bibitem{Lederman} L.M.\ Lederman y D.\ Teresi,
``The God Particle'', Dell Publishing, 1993.

\bibitem{Guardian} Entrevista con Peter Higgs, publicado en
  {\em The Guardian}, Dec. 6, 2013.
  
\bibitem{ATLASCMS} ATLAS Collaboration,
% Observation of a new particle in the search for the Standard Model
% Higgs boson with the ATLAS detector at the LHC
{\em Phys.\ Lett.}\ {\bf B716} (2012) 1-29.
% https://doi.org/10.1016/j.physletb.2012.08.020
CMS Collaboration,
% Observation of a new boson at a mass of 125 GeV with the
% CMS experiment at the LHC".
{\em Phys.\ Lett.}\ {\bf B716} (2012) 30-61.
% doi:10.1016/j.physletb.2012.08.021

\bibitem{neutrinos} A.\ Aguilar-Ar\'{e}valo y W.\ Bietenholz,
% NEUTRINOS: Mysterious Particles with Fascinating
% Features, which led to the Physics Nobel Prize 2015  
{\em Rev.\ Cub.\ F\a'{\i}s.}\ {\bf 32} (2015) 127-136
[versi\'on m\'as extensa: arXiv:1601.04747 [physics.pop-ph]].

\bibitem{HiggsBol} D.\ Ayala Garc\'ia y W.\ Bietenholz,
% Partícula de Higgs: ¿Qué es, y por qué la necesitamos tanto?
% {\em Bolet\'in de la Sociedad Mexicana de F\'isica}
{\em Bol.\ Soc.\ Mex.\ F\'is.}\ {\bf 26-3} (2012) 161-166.  
W.\ Bietenholz,
% The Higgs Particle: what is it, and why did it lead to a
% Nobel Prize in Physics?
{\em Rev. Cub.\ F\'is.}\ {\bf 30} (2013) 109-112.

\bibitem{WBparti}  W.\ Bietenholz,
% What are Elementary Particles? From Dark Energy
% to Quantum Field Excitations
{\em Rev.\ Cub.\ F\a'{\i}s.}\ {\bf 37} (2020) 146-151.
% arXiv:2011.07719 [physics.pop-ph]

\bibitem{Shifman} M.\ Shifman (editor), ``Standing Together in Troubled
  Times: Unpublished Letters by Pauli, Einstein, Franck and Others'',
  World Scientific, 2017.

\bibitem{Nambu} Y.\ Nambu,
% Axial Vector Current Conservation in Weak Interactions
{\em Phys.\ Rev.\ Lett.}\ {\bf 4} (1960) 380-382.
% https://doi.org/10.1103/PhysRevLett.4.380
Y.\ Nambu y G.\ Jona-Lasinio,
% Dynamical Model of Elementary Particles Based on an Analogy
% with Superconductivity. I
{\em Phys.\ Rev.}\ {\bf 122} (1961) 345-358;
% https://doi.org/10.1103/PhysRev.122.345
% Dynamical Model of Elementary Particles Based on an Analogy
% with Superconductivity. II
{\em Phys.\ Rev.}\ {\bf 124} (1961) 246-254.
% https://doi.org/10.1103/PhysRev.124.246

\bibitem{Goldstone} J.\ Goldstone,
% Field Theories with Superconductor Solutions
{\em Nuovo Cim.}\ {\bf 19} (1961) 154-164.
% 10.1007/BF02812722. 
  
\bibitem{GSW} J.\ Goldstone, A.\ Salam y S.\ Weinberg,
% Broken Symmetries
  {\em Phys.\ Rev.}\ {\bf 127} (1962) 965-970.
% https://doi.org/10.1103/PhysRev.127.965

\bibitem{Landau} L.D.\ Landau, 
% On the angular momentum of a system of two photons
{\em Dokl.\ Akad.\ Nauk.\ SSSR} {\bf 60} (1948) 207-209.

\end{thebibliography}

\end{document}

\newpage

\begin{itemize}

\item El mundo consiste de part\'iculas elementales,
  indivisibles, pocos tipos (como 25, depende como se cuenta), ejemplos
  famosos: electr\'on, fot\'on (part\'icula de la luz)
  
\item Descripci\'on relativista con ``campos'', omnipresentes en el
  universo, estado base: vac\'io, excitaciones cu\'antizadas:
  part\'iculas elementales. Un campo para cada tipo de part\'icula,
  excitaciones pueden moverse (como ondas), interactuar,
  generar y destruir part\'iculas
  
\item Concepto central: simetr\'ias, invarianza bajo ciertas
  transformaciones de los campos.

  \subitem global: misma transformaci\'on de un campo de todas lados
  (como gimn\'asia colectiva)

  \subitem local: transformaci\'on independiente en cada punto del
  espacio y tiempo (gimn\'asia ca\'otica)

  Muchas m\'as opciones: simetr\'ia m\'as dif\'icil y poderosa/restrictiva

  Campo adicional, ``de norma'', compensa cambio relativo en
  puntos cercanos en transformaci\'on simultanea, transmite
  interacci\'on: concepto muy exitoso para describir interacciones,
  pero simetr\'ia local tiene que ser exacta

\item Una clase de part\'iculas es conocida como ``fermiones'':
  esp\'in 1/2 (grado de liberad interno, se manifesta como
  momento angular):
  electr\'on y dos primos m\'as pesados, el neutrino (mucho
  m\'as ligero y sin carga el\'ectrica), cuarks (consitutuyentes
  del prot\'on, neutr\'on etc.)

\item Un fermi\'on tiene dos variantes, ``quiralidad'' izquierda o
  derecha (como manos, pero en un sentido abstracto).

  Sin masa: {\it e.g.} electr\'{o}n izquierdo y derecho son
  independientes, pero el t\'ermino para la masa requiere de su producto
  (manos juntas). Simetr\'ia: ambos tienen que transformar de la misma
  manera, pero la teria electrod\'ebil (Glashow, 1961) no es as\'i:
  hay transformaciones que solamente afectan $e_{L}$, pero $e_{R}$ no.

  Incompatible con $m_e$? Pero $m_e \simeq 0.5 \, {\rm MeV}$

  Part\'iculas que transmiten fuerza d\'ebil ($\to$ decaimiento
  radioactivo): $W^{\pm}$, $Z$ (por carga el\'ectrica $\pm1, \ 0$)
  tambi\'en tienen masa aparamente prohibida, pero son del alcance
  muy corta (como $10^{-17}$~m) $\to$ s\'i tienen que ser pesados

  Objeci\'on de Pauli en un seminario en Princeton:
  pregunta matadora
  %de C.N.\ Yang de este tipo de campos
  %de norma en Princeton, dej\'o Yang desesperado
  
\item Mecanismo de Higgs: agrega campo nuevo, ``de Higgs'' $\phi$
  (``escalar'': esp\'in 0)

  Forma producto de $e_L$, $\phi$ y $e_R$; $\phi$ tambi\'en transforma
  bajo simetr\'ia local, t\'ermino total {\em es} invariante

  Baja energ\'ia: campo de Higgs ``se congela'' en su estado base:
  requiere un valor $\neq 0$ para actuar como masa $m_e$ (y
  tambi\'en $M_W$ y $M_Z$)

  Se puede lograr con un potencial tipo sombrero: plano de Higgs,
  altura: valor del potencial, m\'inima con $\phi \neq 0$

  Elige {\it un} m\'inimo: fluctuaciones ($=$ part\'iculas)

  \subitem radiales: resistencia, masa$^{2} \sim $ curvatura del potencial

  \subitem tangencial: masa $=0$, bosones Nambu-Goldstone (NGB), no
  observado en este contexto, problema con Teorema de Goldstone (Coleman)

\item Sin embargo:\\ Observaci\'on de Brout, Englert, Higgs:
  NGB solo para rompimiento de simetr\'ia global.

  Simetr\'ia local: m\'inimos conectados por transformaciones locales,
  fisicamente equivalentes, mismo estado, no hay NGBs.
  Pero las particulas $W$, $Z$ adquieren masa, ``se comen'' los NGBs.
  $W, Z$: interacci\'on d\'ebil de corto alcance, pero el fot\'on
  se queda sin masa: fuerza electromagnetica de largo rango

  Tambi\'en electr\'on y sus primos obtienen masas de esta manera,
  igual que los cuarks

\item El campo de Higgs tiene 4 componentes: 3 tangenciales
  que ``se comen'' las particulas $W^{\pm}$ y $Z$. Se queda la
  componente radial, con masa: la part\'icula de Higgs, tendr\'a
  que ser observable

  Teor\'ia no predice su masa, solamente se pueden
  poner cuotas, carrera por su b\'usqueda era ``en la oscuridad''

  De todas maneras: muy pesado, $> 100$~GeV, creaci\'on requiere colisiones
  de altas energ\'ias. Vive muy poco, $10^{-22}~ {\rm sec}$, no deja
  trayectorias en ning\'un dedector.

  Pero se pueden capturar los productos de su decaimiento,
  en particular $\phi \to 2 \gamma$: se puede reproducir su masa,
  125 GeV, esp\'in 0 (por ejemplo, part\'icula de esp\'in 1 no
  puede decaer en 2$\gamma$, Teorema de Landau) %y Yang
  
  Estudio de muchos decaimientos diferentes, independiente
  en ATLAS y CMS

\item Neutrinos tienen un papel especial: ME tradicional:
   $m_{\nu}=0$, solamente $\nu_L$.

  Consistente; pero finales del siglo 20 se observ\'o: $m_{\nu}>0$

  Parece que requiere $\nu_R$, esto permite mecanismo Higgs para $\nu$,
  y adem\'as otro tipo de masa, solamente de $\nu_R$ (Majorana)

  Sin embargo: $\nu_R$ no observado, y existencia no es seguro:
  t\'ermino de masa unicamente de $\nu_L$, no renormalizable,
  ser\'ia la alternativa

\item Problema de la jerarqu\'ia: parece natural que efectos
  - o correciones - cu\'anticos suben $m_{\rm H}$ a un valor muy alto,
  la ``escala de Planck'' $10^{17}$ veces su valor real.

  Es posible con una ``masa desnuda'' (antes de considerar
  correciones cu\'anticas)
  muy negativo, cancela casi todo hasta un resto diminuto,
  pero esto no parece natural, ``problema de la jerarqu\'ia''
  (proyecto de un estudiante mio: tratamiento con regularizaci\'on
   sofisticada para ``domar'' el efecto) 

\item Masas de part\'iculas elementales vienen del mecanismo de Higgs
  (menos tal vez neutrinos, y otros como el fot\'on se quedan
  sin masa)

  Sin embargo: la masa de un objeto macrosc\'opico de nuestra vida
  cotidiana viene solamente $\sim 1 \dots 2 \%$ del mecanismo de Higgs:
  principalmente masas de nucleones (protones y neutrones) que
  consiste por gran parte de energ\'ia de ``gluones'' (otras part\'iculas
  de norma, que transmiten la interacci\'on fuerte): tienen masa 0,
  pero son confinados al interior de un nucle\'on, y su energ\'ia se
  manifiesta como casi toda la masa del nucleon.

  ?`C\'omo sabemos esto? Tal vez tema para otra charla $\dots$
  
\end{itemize}

\end{document}