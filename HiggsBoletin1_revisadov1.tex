\documentclass[12pt]{article}

\usepackage{geometry}
%\usepackage{showframe} %This line can be used to clearly show the new margins
%\newgeometry{vmargin={15mm}, hmargin={12mm,17mm}}   % set the margins
%\newgeometry{vmargin={20mm}, hmargin={20mm,20mm}}   % set the margins
\newgeometry{vmargin={20mm}, hmargin={20mm,20mm}}

\begin{document}

\begin{center}

{\Large{\underline{\bf En memoria de Peter Higgs}}} \vspace*{2mm} \\
%{\Large{\underline{\bf Peter Higgs: un obituario}}} \vspace*{2mm} \\

\end{center}

\noindent
Peter Higgs fue un f\'isico te\'orico brit\'anico, famoso por su
trabajo de 1964 donde propuso un mecanismo que puede generar masas
para part\'iculas elementales, conforme a la simetr\'{\i}a de norma.
Medio siglo despu\'es, dos experimentos del CERN confirmaron
que este mecanismo est\'a realizado en la naturaleza.
El 8 de abril nos lleg\'{o} la triste noticia del fallecimiento del
gran pionero de la f\'{\i}sica de part\'{\i}culas elementales.
Este art\'{\i}culo es dedicado a su memoria, al mecanismo
y la part\'icula que llevan su nombre.

\section{Datos biogr\'aficos y contexto hist\'orico}

Peter Higgs nació en 1929 en Newcastle, Inglaterra, 
pas\'{o} su juventud parcialmente durante la Segunda Guerra Mundial,
circunstancias que complicaron un poco su formaci\'on escolar.
Despu\'es de la guerra, estudi\'o en Londres, primero matem\'aticas,
luego f\'isica. En 1954, a los 25 a\~nos, ya había terminado su doctorado en el King's College.

Luego trabaj\'{o} temporalmente en la Universidad de Edimburgo,
en el University College y en el Imperial College, ambos en Londres.
En 1960 regres\'{o} a Edimburgo -- ciudad que le encant\'o y donde
el hab\'ia llegado por primera vez en 1949, como estudiante viajando de mochilazo -- para ocupar un puesto de catedr\'{a}tico y quedarse
ah\'{\i} toda su vida.

En 1964, a los 35 a\~{n}os, había escrito sus dos art\'{\i}culos famosos
(y otro sobre el mismo tema en 1966) \cite{Higgs} que llamaron la atenci\'{o}n
y condujeron a invitaciones para presentar seminarios en Princeton
y Harvard. Ten\'{\i}a que defenderse contra audiencias crit\'{\i}cas,
Sidney Coleman coment\'{o} m\'{a}s tarde que en Harvard ``quer\'ian
romper en pedazos al idiota que pensaba que pod\'ia evadir el
Teorema de Goldstone'' \cite{boson}. Result\'{o} que su concepto siguío
de pie, pero aún sin aplicaci\'{o}n fenomenol\'{o}gica
(sus art\'iculos trataron con un modelo de juguete). Adem\'{a}s
se enter\'{o} por Yoichiro Nambu (el árbitro de su primer art\'iculo)
de un trabajo parecido \cite{EB}, publicado
15 d\'ias antes del primer art\'iculo de Higgs sobre el tema.
Los autores eran Fran\c{c}ois Englert y Robert Brout quienes
trabajaron en Bruselas, B\'elgica. Dos meses despu\'{e}s apareci\'{o}
otro art\'{\i}culo relacionado, escrito en Londres por
Gerald Guralnik, Carl Hagen y Tom Kibble \cite{GHK}, pero ellos ya habían
conocido y citaron los trabajos de Englert, Brout y Higgs.

El mecanismo que estos tres art\'{\i}culos propusieron no era
totalmente nuevo: había sido descubierto en 1962/3 en el contexto de la
materia condensada por Philip Anderson \cite{Anderson}. \'El aplic\'o
conceptos de Julian Schwinger \cite{Schwinger} para la explicaci\'on
teórica de la masa de una part\'icula de norma a la teor\'ia de
superconductores. Englert, Brout y Higgs presentaron una
extensi\'on a modelos relativistas.

Totalmente independientemente, en Mosc\'{u} en 1965, dos chicos de 19
a\~nos, ambos de nombre Alexander (o Sasha), con apellidos Migdal y
Polyakov, discutieron en gran detalle qu\'e significa el rompimiento
de una simetr\'ia \cite{Nobel13}. Ellos escribieron otro art\'iculo
parecido que fue publicado en 1966 \cite{MigPol}. Hace unos
a\~{n}os Migdal
visit\'o M\'exico para un evento organizado por Alexander Turbiner,
donde relat\'o sobre las dificuldades que ten\'{\i}an para publicar
este art\'iculo, ya qu\'e la comunidad de f\'{\i}sicos establecidos
en la Uni\'on Sovi\'etica -- bajo el liderazgo de Lev Landau -- rechaz\'o
la teor\'{\i}a cu\'antica de campos, que todav\'{\i}a era muy
controversial tambi\'en en el mundo occidental. Finalmente, este art\'iculo fue publicado con un poco de retraso, pero m\'as tarde ambos Sashas se hicieron
famosos por otros trabajos --- en especial, Polyakov  quien es conocido por
descubrir excitaciones topol\'ogicas que llamamos ahora instantones.

La explicaci\'on de cómo  las part\'iculas de norma -- y ciertas part\'iculas
acopladas -- pueden tener masa fue pronto conocida como el {\em mecanismo
de Higgs},\footnote{Consultar la literatura original no conduce a
  una explicaci\'on clara qu\'e es realmente la raz\'on porqu\'e la
  terminolog\'ia excluye a Brout y Englert, menos tal vez por razones
  fon\'eticas.} el tema de la Secci\'on 2 de este resumen.
Su aplicaci\'on a la fenomenolog\'ia de part\'iculas elementales 
emergi\'o en 1967/8 por parte de Steven Weinberg \cite{Weinberg} y
de Abdus Salam \cite{Salam}.
Ellos integraron este mecanismo al modelo de la interacci\'on
electrod\'ebil que Sheldon Glashow hab\'ia propuesto en 1961
\cite{Glashow} durante su estancia en Copenhague.
De hecho, Glashow estuvo presente en el seminario de Higgs en Harvard,
y reconoci\'o que era ``a nice model'' \cite{boson}, pero nunca se le
occuri\'o la idea de que este mecanismo podr\'ia ser el remedio
para salvar a su modelo, que \'el hab\'ia abandonado.

Sin embargo, esta teor\'ia -- ahora conocida como el sector
elecrod\'ebil de Modelo Estándar -- aún no era generalmente aceptada
por la comunidad ya que era considerada no renormalizable.
Las teor\'ias cu\'anticas de campos casi siempre parecen divergentes
a altas energ\'ias, as\'i que requieren una ``regularizaci\'on'',
una manipulaci\'on matem\'atica que convierte la divergencia
en un valor finito. Se dice que una teor\'ia es {\em renormalizable}
si, al final del c\'alculo, se puede remover la regularizaci\'on
totalmente y llegar a predicciones finitas para las observables
(esta definici\'on es ligeramente simplista).

La comunidad f\'isica cambi\'o su punto de vista en 1971/2, gracias
al trabajo de Gerard 't Hooft, un brillante estudiane de doctorado
en Utrecht, Holanda, quien present\'o evidencia a favor de la
renormalizibilidad de dicho modelo (parcialmente junto a
su asesor, Martinus Veltman). Estos trabajos \cite{tHooft} fueron
una sensaci\'on que causaron un cambio de paradigma en aquella \'epoca.

La clave para este hito fue un nuevo m\'etodo, la {\em regularizaci\'on
dimensional} que es entre los logros principales de la f\'isica en
Am\'erica Latina: fue propuesta primero por dos argentinos, Carlos Bollini
y Juan Jos\'e Giambiagi en 1971, aunque la publicaci\'on \cite{BolGiam}
se demor\'o hasta 1972. Ellos trabajaron en La Plata, en
circunstancias dif\'iciles durante la dictadura militar \cite{DimReg}.

Agregamos que hoy en d\'ia se da menos importancia a la pregunta
si el Modelo Estándar es renormalizable: la tendencia es que se
considera como teor\'ia efectiva, y su validez en un gran rango
energ\'etico -- que no tiene que extenderse hacia infinito --
es suficiente.

Poco despu\'es, en 1973, el Modelo Estándar de las part\'iculas
elementales fue completamente establecido, con un sector
electrod\'ebil \cite{Weinberg,Salam} y otro de la interacci\'on
fuerte \cite{QCD}. En ambos sectores,
el mecanismo de Higgs es indispensable para proporcionar masas
a gran parte de las part\'iculas elementales.
Esto fue una revoluci\'on en la f\'isica de altas energ\'ias como no la
hemos visto m\'as en el medio siglo que sigui\'o, aunque después el progreso
fue relativamente lento.

En el siglo XXI es popular especular sobre f\'isica m\'as all\'a
del Modelo Estándar. Sin embargo, por ahora ninguna de estas propuestas
tiene apoyo sólido de datos experimentales. Por otro lado, los
experimentos han confirmado las predicciones del Modelo Estándar una
y otra vez: muchas veces salien al aire en las noticias que el Modelo Estándar ha sido ``refutado''
por nuevos resultados, pero al final del an\'alisis, y la repetici\'on
de los experimentos, siempre sus predicciones han sido
aprobadas.\footnote{Como ejemplo reciente, en la segunda parte de la
  decada pasada, se difudieron noticas de una tensi\'on entre el Modelo
  Estandar y experimentos con el decaimiento de mesones pesados
  conocidos como ``mesones B''. Al final, esta discrepacia no se
  substanci\'o. La \'ultima moda es el momento magn\'etico del
  muón, donde el valor experimental parece un poquito diferente
  del c\'alculo basado en el Modelo Estándar. Si esto es verdad --
  cosa que no es nada certera -- la predicci\'on se equivoca al
  nivel relativo de $10^{-10}$: si lo comparamos con la distancia
  entre M\'exico y Europa central (Suiza por ejemplo), unos
  10,000\,km, esto corresponde a un posible error de la magntitud
  de milímetros. Pero cálculos con simulaciones num\'ericas
  en la ret\'icula conducen a resultados m\'as cercanos al valor
  experimental, tendencia que indica que incluso esta discrepancia
  m\'inima podr\'ia desaparecer con un an\'alisis m\'as preciso,
  igual que todas las supuestas discrepancias anteriores.}
  
El Modelo Estándar es aún incompleto para describir al universo (faltan por ejemplo
la gravitaci\'on, materia oscura y energ\'ia oscura), pero
a\'un as\'i: se trata de nada menos que la teor\'ia m\'as precisa
y -- en este sentido -- m\'as exitosa en la historia de la ciencia.\\
 
Higgs ya no particip\'o en este desarollo r\'apido. \'El ya era
tan famoso que pod\'ia permitirse casi no publicar m\'as resultados
de investigaci\'on a partir de la edad de 40 a\~{n}os.
(En M\'exico esto ser\'ia un problema s\'erio con el SNII etc.)
 
Era conocido como persona tranquila y modesta, casi t\'imida,
que no busc\'o la atenci\'on medi\'atica o ponerse en el centro de atenci\'on en eventos. Con su mentalidad de abstenerse del
``show'', se puede caracterizar como el contrario a Feynman.
Esta caracterizaci\'on corresponde a la impresi\'on de uno de los
autores (WB) qui\'en particip\'o en un congreso en Edimburgo 1997.
Higgs (quien era em\'erito desde 1996) apareci\'o en el banquete,
pero muy discreto, simplemente para sentarse en una mesa sin
ning\'un espect\'aculo.

Esto no significa que Higgs no ten\'ia convicciones: era
temporalmente activista para el desarmamento nuclear y
por el movimiento ambiental como miembro de Green Peace. \\

Una vez que el Modelo Estándar había sido establecido, su exploraci\'on
progresó con trabajo intenso en m\'ultiples pa\'ises.
En el a\~no 2000, todas sus part\'iculas ya eran experimentalmente
encontradas, menos una: la famosa ``part\'icula de Higgs'', involucrada
en este mecanismo, como vamos a descibir en la Secci\'on 2.

Otra vez, la nomenclatura es tal vez un poco injusta con Englert y
Brout, pero as\'i es la convenci\'on de la comunidad.
Higgs no puso este t\'ermino, pero tambi\'en le era incómodo el
apodo absurdo ``part\'icula de dios'' que no tiene ni el menor sentido.
Este término fue propuesto por la editorial de un libro de divulgaci\'on \cite{Lederman},
obviamente con un objetivo comerical, pero totalmente irresponsable.
Este t\'ermino se hizo popular y condujo a confusi\'on sin fin,
por ejemplo, la iglesia cat\'olica de Espa\~na crey\'o que el trabajo al
CERN tuviera algo que ver con teolog\'ia \cite{boson}. !`Tenemos que
tener cuidado con los terminos que se usamos!

En el siglo XXI fuimos testigos de una carrera emocionante en la
b\'usqueda de la part\'icula de Higgs. En su fase final, era una
compentencia entre el Fermilab (cerca de Chicago) y el CERN
(cernca de Ginebra, sobre la frontera entre Suiza y Francia).
Despu\'es de primeras indicaciones en 2011, en 2012 las
colaboraciones ATLAS y CMS, ambos trabajando de manera independiente
en el Gran Colisionador de Hadrones (LHC) al CERN, presentaron
evidencia para la observaci\'on indirecta pero clara de la
part\'icula de Higgs qu\'e era tanto buscada \cite{ATLASCMS}.

Con esto todo el conjunto de part\'iculas del Modelo Estandar
había sido observado. As\'i se confirm\'o que el mecanismo de Higgs
est\'a realizado en la naturalza, 48 a\~nos despu\'es de su
propuesta te\'orica. Esto se demor\'o casi el doble del tiempo
que la observaci\'on del neutrino (predicho por Wolfgang Pauli en 1930,
detectado por Clyde Cowan, Frederick Reines y colaboradores en
1956), vemos que a veces vale la pena tener paciencia.

En particular, valió la pena para Englert y Higgs, quienes recibieron
el Premio Nobel en 2013 por su predicci\'on correcta \cite{Nobel13};
tristemente, Brout hab\'ia muerto poco antes, en 2011.
En abril del 2024 nos lleg\'o la noticia del fallecimiento de Higgs,
a los 94 a\~nos, despu\'es de una breve enfermedad.

\section{Mecanismo de Higgs}

\begin{thebibliography}{10}

\bibitem{Higgs} P.W.\ Higgs,
% Broken symmetries, massless particles and gauge fields,
{\em Phys.\ Lett.}\ {\bf 12} (1964) 132-133;
% doi:10.1016/0031-9163(64)91136-9
% Broken Symmetries and the Masses of Gauge Bosons,
{\em Phys.\ Rev.\ Lett.}\ {\bf 13} (1964) 508-509.
% doi:10.1103/PhysRevLett.13.508
% Spontaneous Symmetry Breakdown without Massless Bosons,
{\em Phys.\ Rev.}\ {\bf 145} (1966) 1156-1163.
% doi:10.1103/PhysRev.145.1156

\bibitem{boson} P.\ Higgs,
% My Life as a Boson: The Story of ``The Higgs'',
{\em Int.\ J.\ Phys.}\ {\bf A17} (2002) 86-88.
% https://doi.org/10.1142/S0217751X02013046

\bibitem{EB} F.\ Englert y R.\ Brout,
% Broken Symmetry and the Mass of Gauge Vector Mesons,
{\em Phys.\ Rev.\ Lett.}\ {\bf 13} (1964) 321-323.
% doi:10.1103/PhysRevLett.13.321

\bibitem{GHK} G.S.\ Guralnik, C.R.\ Hagen y T.W.B.\ Kibble,
% Global Conservation Laws and Massless Particles,
{\em Phys.\ Rev.\ Lett.}\ {\bf 13} (1964) 585-587.
% doi:10.1103/PhysRevLett.13.585

\bibitem{Anderson} P.W.\ Anderson,
% Plasmons, Gauge Invariance, and Mass,
{\it Phys.\ Rev.}\ {\bf 130} (1963) 439-442.
% doi:10.1103/PhysRev.130.439

\bibitem{Schwinger} J.S.\ Schwinger,
% Gauge Invariance and Mass,
  {\em Phys.\ Rev.}\ {\bf 125} (1962) 397-398;
% doi:10.1103/PhysRev.125.397
% Gauge Invariance and Mass. 2.,
{\em Phys.\ Rev.}\ {\bf 128} (1962) 2425-2429.
% doi:10.1103/PhysRev.128.2425

\bibitem{Nobel13} Royal Swedish Academy of Sciences,
  ``Scientific Background: The BEH-Mechanism, Interactions with Short
  Range Forces and Scalar Particles'',\\
  www.nobelprize.org/prizes/physics/2013/advanced-information/
  
\bibitem{MigPol} A.A.\ Migdal y A.M.\ Polyakov,
% SPONTANEOUS BREAKDOWN OF STRONG INTERACTION SYMMETRY
% AND THE ABSENCE OF MASSLESS PARTICLES,
{\em Zh.\ Eksp.\ Teor.\ Fiz.}\ {\bf 51} (1966) 135-146 
[{\em Sov.\ Phys.\ JETP} {\bf 24} (1967) 91-98].

\bibitem{Weinberg} S.\ Weinberg,
% A Model of Leptons
{\em Phys.\ Rev.\ Lett.}\ {\bf 19} (1967) 1264-1266.
% doi:10.1103/PhysRevLett.19.1264  

\bibitem{Salam} A.\ Salam,
% Weak and Electromagnetic Interactions,
  en {\em Proc.\ of the 8th Nobel Symposium on ``Elementary
  particle theory, relativistic groups and analyticity''},
  editor N.\ Svartholm (1968) p.\ 367-377.
% Conf. Proc. C \textbf{680519}, 367-377 (1968)
% doi:10.1142/9789812795915\_0034

\bibitem{Glashow} S.L.\ Glashow,
% Partial Symmetries of Weak Interactions
{\it Nucl.\ Phys.}\ {\bf 22} (1961) 579-588.
% doi:10.1016/0029-5582(61)90469-2

\bibitem{tHooft} G.\ 't Hooft,
% Renormalization of Massless Yang-Mills Fields
{\em Nucl.\ Phys.}\ {\bf B33} (1971) 173-199;
% doi:10.1016/0550-3213(71)90395-6
% Renormalizable Lagrangians for Massive Yang-Mills Fields
{\em Nucl.\ Phys.}\ {\bf B35} (1971) 167-188.
G.\ 't Hooft y M.J.G.\ Veltman,
% doi:10.1016/0550-3213(71)90139-8
% Regularization and Renormalization of Gauge Fields,
{\em Nucl.\ Phys.}\ {\bf B44} (1972) 189-213.
% doi:10.1016/0550-3213(72)90279-9

\bibitem{BolGiam} C.G.\ Bollini y J.J.\ Giambiagi,
% Dimensional Renormalization: The Number of Dimensions
% as a Regularizing Parameter,
{\em Nuovo Cim.}\ {\bf B12} (1972) 20-26;
% doi:10.1007/BF02895558
% Lowest order divergent graphs in nu-dimensional space,
{\em Phys.\ Lett.}\ {\bf B40} (1972) 566-568.
% doi:10.1016/0370-2693(72)90483-2

\bibitem{DimReg} W.\ Bietenholz y L.\ Prado, 
% 40 Years of Calculus in 4+ epsilon Dimensions
%{\em Bolet\'in de la Sociedad Mexicana de F\'isica}
{\em Bol.\ Soc.\ Mex.\ F\'is.}\ {\bf 26-4} (2012) 227-230;
% Revolutionary physics in reactionary Argentina
{\em Physics Today} {\bf 67} (2014) 38-43.
% DOI 10.1063/PT.3.2277

\bibitem{QCD} H.\ Fritzsch, M.\ Gell-Mann y H.\ Leutwyler,
% Advantages of the Color Octet Gluon Picture,
{\em Phys.\ Lett.}\ {\bf B47} (1973) 365-368.
% doi:10.1016/0370-2693(73)90625-4

\bibitem{Lederman} L.M.\ Lederman y D.\ Teresi,
``The God Particle'', Dell Publishing, 1993.

\bibitem{Guardian} Entrevista con Peter Higgs, publicado en
  {\em The Guardian}, Dec. 6, 2013.
  
\bibitem{ATLASCMS} ATLAS Collaboration,
% Observation of a new particle in the search for the Standard Model
% Higgs boson with the ATLAS detector at the LHC
{\em Phys.\ Lett.}\ {\bf B716} (2012) 1-29.
% https://doi.org/10.1016/j.physletb.2012.08.020
CMS Collaboration,
% Observation of a new boson at a mass of 125 GeV with the
% CMS experiment at the LHC".
{\em Phys.\ Lett.}\ {\bf B716} (2012) 30-61.
% doi:10.1016/j.physletb.2012.08.021

\bibitem{neutrinos} A.\ Aguilar-Ar\'{e}valo y W.\ Bietenholz,
% NEUTRINOS: Mysterious Particles with Fascinating
% Features, which led to the Physics Nobel Prize 2015  
{\em Rev.\ Cub.\ F\a'{\i}s.}\ {\bf 32} (2015) 127-136
[versi\'on m\'as extensa: arXiv:1601.04747 [physics.pop-ph]].

\bibitem{YouTube} 
www.youtube.com/watch?v=E$\_$tFrbnKYto\&list=LL\&index=1
  
\bibitem{HiggsBol} D.\ Ayala Garc\'ia y W.\ Bietenholz,
% Partícula de Higgs: ¿Qué es, y por qué la necesitamos tanto?
% {\em Bolet\'in de la Sociedad Mexicana de F\'isica}
{\em Bol.\ Soc.\ Mex.\ F\'is.}\ {\bf 26-3} (2012) 161-166.  
W.\ Bietenholz,
% The Higgs Particle: what is it, and why did it lead to a
% Nobel Prize in Physics?
{\em Rev. Cub.\ F\'is.}\ {\bf 30} (2013) 109-112.

\bibitem{WBparti}  W.\ Bietenholz,
% What are Elementary Particles? From Dark Energy
% to Quantum Field Excitations
{\em Rev.\ Cub.\ F\a'{\i}s.}\ {\bf 37} (2020) 146-151.
% arXiv:2011.07719 [physics.pop-ph]

\bibitem{Shifman} M.\ Shifman (editor), ``Standing Together in Troubled
  Times: Unpublished Letters by Pauli, Einstein, Franck and Others'',
  World Scientific, 2017.

\bibitem{Nambu} Y.\ Nambu,
% Axial Vector Current Conservation in Weak Interactions
{\em Phys.\ Rev.\ Lett.}\ {\bf 4} (1960) 380-382.
% https://doi.org/10.1103/PhysRevLett.4.380
Y.\ Nambu y G.\ Jona-Lasinio,
% Dynamical Model of Elementary Particles Based on an Analogy
% with Superconductivity. I
{\em Phys.\ Rev.}\ {\bf 122} (1961) 345-358;
% https://doi.org/10.1103/PhysRev.122.345
% Dynamical Model of Elementary Particles Based on an Analogy
% with Superconductivity. II
{\em Phys.\ Rev.}\ {\bf 124} (1961) 246-254.
% https://doi.org/10.1103/PhysRev.124.246

\bibitem{Goldstone} J.\ Goldstone,
% Field Theories with Superconductor Solutions
{\em Nuovo Cim.}\ {\bf 19} (1961) 154-164.
% 10.1007/BF02812722. 
  
\bibitem{GSW} J.\ Goldstone, A.\ Salam y S.\ Weinberg,
% Broken Symmetries
  {\em Phys.\ Rev.}\ {\bf 127} (1962) 965-970.
% https://doi.org/10.1103/PhysRev.127.965

\bibitem{Landau} L.D.\ Landau, 
% On the angular momentum of a system of two photons
{\em Dokl.\ Akad.\ Nauk.\ SSSR} {\bf 60} (1948) 207-209.

\end{thebibliography}

\end{document}
